1. Smartphone is easily lost or stolen by an attacker as its small size. This may result in privacy leakage and finance lost. It is reported by lookout.com that nearly \$2.5 billion worth of devices were lost or stolen in 2011~\cite{lookout-survey}.

2. Human face plays an important role in our social interaction. As compared with other biometric modalities such as fingerprint and iris, face recognition has distinct advantages because of its non-contact process. Face images could be captured from a distance without touching person being identified and identification does not require interaction with person. In addition, face recognition serves crime deterrent purpose because face images that have been recorded and archived could later help identify a person.

3. Human identification system based on biometrics other than the face have already led to commercial products with very high identification rates: the iris~\cite{daugman1993high} and fingerprints~\cite{rogers1994biometric} can be cited as example. However, these systems are not always appreciated by users, as they require some close interaction with the machine often perceived as invasive. Moreover, they require the user to stop at the device and be cooperative, which is acceptable for access control to restricted areas, but not for other applications like surveillance. Face recognition may overcome some of these limitations.

4. Challenges in face recognition arise because the face is not a rigid object and images can be taken from many different viewpoints of the face.

5. Biometric-based techniques have emerged as the most promising option for recognizing individuals in recent years since, instead of authenticating people and granting them access to physical and virtual domains based on passwords, PINs, smart cards, plastic cards, tokens, keys and so forth, thses methods examine an individual's physiological and/or behavioral characteristics in order yo determine and/or ascertain his identity. Passwords and PINs are hard to remember and can be stolen or guessed; cards, tokens, keys and the like can be misplaced, forgotten, purloined or duplicated; magnetic cards can become corrupted and unreadable. However, an individual's biological traits cannot be misplaced, forgotten, stolen or forged.

Biometric-based technologies include identification based on physiological characteristics (such as face, fingerprints, finger geometry, hand geometry, hand veins, palm, iris, retina, ear and voice) and behavioral traits (such as gait, signature and keystroke dynamics). Face recognition appears to offer several advantages over other biometric methods as follows:

\begin{itemize}
    \item Almost all these technologies require some voluntary action by the user, i.e., the user needs to place his hand on a hand-rest for fingerprinting or hand geometry detection and has to stand in a fixed position in front of a camera for iris or retina identification. However, face recognition can be done passively without any explicit action or participation on the part of the user since face images can be acquired from a distance by a camera. This is particulary beneficial for security and surveillance purposes.
    \item Data acquisition in general is fraught with problems for other biometrics: techniques that rely on hands and fingers can be rendered useless if the epidermis tissue is damaged in some way (i.e., bruise or cracked). Iris and retina identification require expensive equipment and are much too sensitive to any body motion. Voice recognition is susceptible to background noises in public places and auditory fluctuations on a phone line or tape recording. Signatures can be modified or forged. However, facial images can be easily obtained with a couple of inexpensive fixed cameras.
    \item Good face recognition algorithms and appropriate preprocessing of the images can compensate for noise and slight variations in orientation, scale and illumination.
    \item Technologies that require multiple individuals to use the same equipment to capture their biological characteristics potentially expose the user to the transmission of germs and impurities from other users. However, face recognition is totally non-intrusive and does not carry any such health risks.
\end{itemize}

6. In this section, we discuss the technical considerations that underpin the design of CarSafe while the detailed design is presented in section 3.  CarSafe fundamentally relies on the real-time processing of dual camera video streams. In what follows, we discuss the challenges and design considerations that arise.  This example motivates the need for the simultaneously processing of video streams from the front and rear cameras. SURF features are provided to a binary two-class SVM that is trained to classify eyes as either being open or closed (defined as an open or closed event).

7. Related Work--Face recognition techniques can be broadly divided into three categories based on the face data acquisition methodology: methods that operate on intensity images; those that deal with video sequences; and those that require other sensory data such 3D information or infra-red imagery~\cite{Jafri2009A}.

8. Among our interesting finding is how large a role web passwords play in user lives. The average user has 6.5 passwords, each of which is shared across 3.9 different sites. Each user has about 25 accounts that require passwords, and types an average of 8 passwords per day~\cite{florencio2007}.

9. With the advances in miniaturization techniques, performance of the mobile and portable devices is rapidly increasing. This enables to use such devices not only as communication tools but also in an applications like m-banking~\cite{Pousttchi2004Assessment} or m-government~\cite{Kim1970Architecture}. This means that they can store and process valuable information such as financial or private data. According to UK statistics in every three minutes a mobile phone is stolen~\cite{hugeSurge2006}. The current protection mechanisms of these devices are usually based on PIN code or passwords. Nowadays a "heavy" user has on average 21 passwords to remember~\cite{2002NATMonitor}. Unfortunately, 81\% of the users select common words as a passwords and 30\% of users write their passwords down, which equally compromises security~\cite{2002NATMonitor}. Recently, biometric modalities such as fingerprints~\cite{Su2005A,Chen2005A} have been proposed for mobile devices. However, both fingerprints and password entry are obtrusive and require explicit action from the user, which is not convenient in a frequent use. In order to improve security in mobile and portable devices, an unobtrusive mechanisms of authentication is desirable.

10. Geometric Deformation Features. The geometric displacement of certain selected Candide node,defined as the difference of the node coordinates between the first and the greatest facial expression intensity frame, is used as an input to a novel multiclass Support Vectot Machine (SVM) system of classifiers that are used to recognize either the six basic facial expressions or a set of chosen Facial Action Units (FAUs).
\textbf{The leave-one cross-validation approach was used in order to make maximal use of the available data and produce averaged classification accuracy results.}~\cite{Kotsia2007Facial}.

11. The benefits of feature selection are not only to reduce recognition time by reducing the amount of data that needs to be analyzed, but also, in many cases, to produce better classification accuracy due to finite sample size effects~\cite{Jain1997Feature}.

12. Whether to deny authentication to the user with respect to accessing one or more functionalities controlled by the computing device. 

Grant authentication to the user with respect to accessing one or more functionalities controlled by the computing device.

Wherein the at least one facial landmark \textbf{comprises at} least one of an eye, an eyebrow, a mouth area, a forehead area, and a nose, and wherein the indicated facial gesture includes at least one of a blink gesture, a wink gesture, an ocular movement, a smile gesture, a frown gesture, a tongue protrusion gesture, an open mouth gesture , an eyebrow movement, a forehead wrinkle gesture, and a nose wrinkle gesture.

A user may activate or otherwise gain access to functionalities controlled by a computing device by "unlocking" the device.

The present disclosure also describe additional techniques to prevent erroneous authentication caused by spoofing.
